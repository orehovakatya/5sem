% !TeX spellcheck = russian-aot
%\documentclass[12pt,a4paper,draft]{article}
%\usepackage{cmap}
%\usepackage[utf8]{inputenc}
%\usepackage[T2A]{fontenc}
%\usepackage[english,german,russian]{babel}
%\usepackage{amsmath}
%\usepackage{amsfonts}
%\usepackage{amssymb}
%\usepackage[final]{graphicx}
%\DeclareGraphicsExtensions{.jpg,.png}
%\graphicspath{{pictures/}} % путь к графическим файлам. Пусть они помещаются в подкаталог pictures текущего каталога
%\usepackage[figurename=Рисунок,labelsep=period]{caption}
%\usepackage{float}
%\usepackage{indentfirst}
%\usepackage[pdftex,left=2.5cm,right=2.5cm,top=3cm,bottom=3cm]{geometry}
%\usepackage[obeyDraft]{todonotes}
%\usepackage[hidelinks,draft=false]{hyperref}
%\frenchspacing
%\pdfcompresslevel=9

\documentclass[a4paper,12pt]{article}
\usepackage[utf8]{inputenc}
\usepackage[T2A]{fontenc}
\usepackage[english,russian]{babel}
\usepackage{natbib}
\usepackage[final]{graphicx}
\DeclareGraphicsExtensions{.jpg,.png}
\graphicspath{{pictures/}}
\usepackage{float}
\usepackage{amsmath}
\usepackage{pgfplots}
\usepackage{color} %% это для отображения цвета в коде
%\usepackage{listings} %% собственно, это и есть пакет listings




%\setmonofont{Consolas} %to be used with XeLaTeX or LuaLaTeX
\definecolor{bluekeywords}{rgb}{0,0,1}
\definecolor{greencomments}{rgb}{0,0.5,0}
\definecolor{redstrings}{rgb}{0.64,0.08,0.08}
\definecolor{xmlcomments}{rgb}{0.5,0.5,0.5}
\definecolor{types}{rgb}{0.17,0.57,0.68}

\usepackage{listings}
\lstset{
language=[Sharp]C,
captionpos=t,
%numbers=left, %Nummerierung
%numberstyle=\tiny, % kleine Zeilennummern
frame=lines, % Oberhalb und unterhalb des Listings ist eine Linie
showspaces=false,
showtabs=false,
breaklines=true,
showstringspaces=false,
breakatwhitespace=true,
escapeinside={(*@}{@*)},
commentstyle=\color{greencomments},
morekeywords={partial, var, value, get, set},
keywordstyle=\color{bluekeywords}\bf,
stringstyle=\color{redstrings},
basicstyle=\ttfamily\small,
extendedchars=\true
}

\usepackage{caption}
\DeclareCaptionFont{white}{\color{white}} %% это сделает текст заголовка белым
%% код ниже нарисует серую рамочку вокруг заголовка кода.
\DeclareCaptionFormat{listing}{\colorbox{gray}{\parbox{\textwidth}{#1#2#3}}}
\captionsetup[lstlisting]{format=listing,labelfont=white,textfont=white}


\usepackage{ctable}%для таблиц
\captionsetup[table]{justification=raggedleft,singlelinecheck=off}




%%%%%%%%%%%%%%%%%%%%%%%%%%%%%%%%%%%%%%%%%%%%%%%%%%%%%%%%%%%%%%%%%%%%%%%%%%%%%%%%%%%%%%%%%%%%%%%

\begin{document}
\begin{titlepage}
	\centering
    \begin{figure}[H]
    	\includegraphics[scale=1.2]{photo}
   	\end{figure}
	{\scshape Министерство образования Российской Федерации
Московский Государственный Технический Университет им. Н.Э. Баумана \par}
	\vspace{4cm}
	{\scshape\Large Отчёт по лабораторной работе № 1\par}
    {\scshape\Large По курсу: "Анализ алгоритмов"\par}
	{\scshape\Large\bf Тема:"Потоковый алгоритм."\par}
    \vspace{2cm}
    {\flushright Студент: Орехова Е.О. ИУ7-51\par
    \flushright Преподаватель: Волкова Л.Л.\par}
    \vspace{2cm}
% Bottom of the page
	{\large \today\par}
\end{titlepage}

\def\contentaname{Содержание}
\tableofcontents %Вывод содержания
\clearpage

\section{Постановка задачи}
    Придумать и реализовать задачу, используя потоковый алгоритм.

\section{Идея}
    Программа считает среднее арифметическое чисел. В программе существует два потока: первый создаёт числа и отправляет их в очередь, второй - берёт числа из очереди и пересчитывает среднее арифметическое.
	
    
\section{Реализация}
\begin{lstlisting}[label=some-code,caption={Производитель чисел}]
static void conv1()
{
	double temp;
	for (int i = 0; i < len; i++)
	{
		temp = 10*rand.NextDouble();//[0-10)
		lock (locker)
		{
			queue1.Enqueue(temp);
			Console.WriteLine("1.Работает: положил: {0} из {1}", i+1, len);
		}
		
	}
}
\end{lstlisting}

\begin{lstlisting}[label=some-code,caption={Вычисление среднего арифметического}]
static void conv2()
{
	double temp;
	while (work)
	{
		if (queue1.Count != 0)
		{
			lock (locker)
			{
				temp = queue1.Dequeue();
				Sum += temp;
				count++;
				midle = Sum / count;
			}
			Console.WriteLine("2.Работает: вычислил {0} осталось {1}. Среднее арифметическое: {2}",count,queue1.Count,midle);
			if (count == len)
				work = false;
		}
	}
}
\end{lstlisting}

\section{Заключение}
	В ходе лабораторной работы был реализован потоковый алгоритм вычисления среднего арифметического чисел.
\end{document} 